\chapter{Discussion} \label{chap:discussion}
\section{Summary of the Results}
For the subsonic flow, it is stable under Dirichlet boundary condition. It is also stable under fixed-open boundary, except for ground mode when velocity is nonuniform. For the supersonic flow, it is stable when Dirichlet boundary condition is applied, and unstable is the boundary condition is fixed-open. The results are summarized in Table.~\ref{table:results-subsonic-supersonic}.

The flow with accelerating velocity profile is stable, assuming there is no velocity perturbation at the entrance of the nozzle. On the other hand, the decelerating flow is physically impossible under the boundary condition where velocity perturbation is 0 at the entrance. The results are summarized in Table.~\ref{table:results-accelerating-decelerating}.

\begin{table} [htbp]
	\centering
	\caption{The Dirichlet boundary condition means there are no perturbations at the two ends of the nozzle. While the fixed-open condition assume that there is no perturbation at the entrance of the nozzle, $\tilde{v}(-1)=0$, and then the derivative with respect to $z$ of the velocity perturbation is 0 at the exit, $\partial_z\tilde{v}(1)=0$.}
	\begin{tabular}{| c | c | c |}
		\hline
		           & Subsonic                                                  & Supersonic \\
		\hline
		Dirichlet  & Stable                                                    & Stable     \\
		\hline
		Fixed-Open & Stable (except ground mode for variable velocity profile) & Unstable   \\
		\hline
	\end{tabular}
	\label{table:results-subsonic-supersonic}
\end{table}

\begin{table} [htbp]
	\centering
	\caption{The boundary condition is set so that there is no perturbation at the entrance of the nozzle, $\tilde{v}(-1)=0$, and then reaches a finite value at the throat of the nozzle, $\tilde{v}(0)=1$.}
	\begin{tabular}{|c | c|}
		\hline
		Accelerating & Decelerating \\
		\hline
		Stable       & Impossible   \\
		\hline
	\end{tabular}
	\label{table:results-accelerating-decelerating}
\end{table}

\section{Limitations of the methods}
\subsection{Spectral Method}
The spectral method suffers the spectral pollution. For now there is no automatic ways to filter spurious modes other than doing convergence test and pick up the convergent eigenvalues manually by ourselves. We believe there is a discretization scheme that is spectral pollution free. We made such optimistic guess based on the following example. Let's transform Eq.~\ref{eq:constant-v-problem-dirichlet} with Dirichlet boundary condition $\tilde{v}(\pm 1) = 0$ to its normal form, that is
\begin{equation}
	u''(z) + r(z)u(z) = 0, \quad u(\pm 1) = 0
\end{equation}
by doing the change of variables
\begin{equation}
	u(z) = \exp\left(\frac{iv_0\omega}{1-v_0^2}\right)\tilde{v}(z); \quad
	r(z) = \frac{\omega^2}{(1-v_0^2)^2}
\end{equation}
This scheme is spectral pollution free. Meaning that no spurious modes will occur if we apply spectral methods to this transformed problem.


\subsection{Shooting Method}
The shooting method is not exhaustive due to the nature of root finding algorithm. A root can only be found if the initial guess of the root is close enough to the actual root. To work around this issue, we have to perform a grid search on the complex plane. This way we can only survey the low frequency region due to finite computing resources. The conclusion for cases with transonic velocity profiles is true only for low frequency region.

\section{Conclusion}
In Chap.~\ref{chap:polynomial-eigenvalue-problem}, we derived the linearized equations of motion of the flow in one dimensional magnetic nozzle. Furthermore, we rewrite the linearized governing equations as an eigenvalue problem. Using the spectral methods introduced in chapter 2, we discretized the operators of the problem. Hence, transforming it into an algebraic eigenvalue problem.

With the aid of computer, we are able to solve the algebraic eigenvalue problem. The results show that the flow in magnetic nozzle with Dirichlet boundary condition is stable except the case with decelerating velocity profile.
