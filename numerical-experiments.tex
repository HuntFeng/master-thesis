\chapter{Numerical Experiments} \label{chap:numerical-experiments}
In this chapter, we will solve the polynomial eigenvalue problem, Eq.~(\ref{eq:polynomial-eigenvalue-problem}), under different boundary conditions.

For subsonic and supersonic velocity profiles, spectral method will be used. Eq.~(\ref{eq:polynomial-eigenvalue-problem}) will be solved under Dirichlet boundary condition and fixed-open boundary condition. The former boundary condition indicates that the perturbation is 0 at the nozzle entrance and exit $\tilde{v}(\pm 1)=0$. The fixed-open boundary condition means $\tilde{v}(-1)=\tilde{v}'(1) = 0$. We will utilize the spectral method to solve the polynomial eigenvalue problem, employing various discretization techniques including finite difference (FD), finite element (FE), and spectral element (SE). This approach ensures the accuracy and reliability of the results. The finite difference method will be used together with equally spaced nodes. The finite element method will use B-spline as basis functions. Finally, the spectral element method uses sine functions as the spectral elements. The parameters of spectral method are summarized in Table.~(\ref{table:parameters-dirichlet}) and Table.~(\ref{table:parameters-fixed-open}).
\begin{table} [H]
	\centering
	\caption{With Dirichlet boundary condition, all methods have good accuracy, so using 101 nodes in the region $[0,1]$ is enough. For FE and SE methods, we use 50 basis functions.}
	\begin{tabular}{|c|c|c|c|}
		\hline
		           & FD  & FE\_BSPLINE & SE\_SINE \\
		\hline
		N          & 101 & 101         & 101      \\
		\hline
		NUM\_BASIS &     & 51          & 50       \\
		\hline
	\end{tabular}
	\label{table:parameters-dirichlet}
\end{table}
\begin{table} [H]
	\centering
	\caption{With fixed-open boundary condition, it requires higher resolution in order to get accurate results. Therefore, all methods use 501 nodes in the region $[0,1]$, and FE method uses 101 basis functions.}
	\begin{tabular}{|c|c|c|}
		\hline
		           & FD  & FE\_BSPLINE \\
		\hline
		N          & 501 & 501         \\
		\hline
		NUM\_BASIS &     & 101         \\
		\hline
	\end{tabular}
	\label{table:parameters-fixed-open}
\end{table}

When dealing with accelerating and decelerating velocity profiles, as outlined in Chap.~\ref{chap:singular-perturbation}, the spectral method struggles to yield meaningful results because of the singularity at the nozzle throat ($z=0$). In such cases, we will resort to the shooting method for resolution. We will set boundary condition to $\tilde{v}(-1) = 0$ and $\tilde{v}(0)=1$.
\section{Subsonic Case}
\subsection{Constant Velocity Profile}
Eq.~(\ref{eq:constant-v-problem-dirichlet}) is a special case of a more general polynomial problem Eq.~(\ref{eq:polynomial-eigenvalue-problem}). The existence of the exact solution allows us to verify the correctness of each method's implementation. This also serves as a reference to the accuracy spectral methods can achieve.

\subsubsection*{Dirichlet Boundary}
From Fig.~\ref{fig:constant-v-dirichlet-subsonic}, we see that the order of growth rates obtained by different methods is about $~10^{-14}$. We will use these numbers as a reference to the accuracy of our numerical methods. If a method produces growth rates with order close to $10^{-14}$, we consider the growth rates to be 0.

\begin{table} [H]
	\centering
	\caption{Relative error of each eigenvalue. Numerical results agree with exact solution well.}
	\begin{tabular}{|c|c|c|c|c|c|}
		\hline
		$v_0=0.5$ & 1         & 2         & 3         & 4         & 5         \\
		\hline
		FD        & 2.827e-05 & 1.130e-04 & 2.541e-04 & 4.512e-04 & 7.040e-04 \\
		\hline
		FE        & 0.005     & 0.005     & 0.006     & 0.008     & 0.010     \\
		\hline
		SE        & 2.896e-05 & 1.157e-04 & 2.603e-04 & 4.626e-04 & 7.217e-04 \\
		\hline
	\end{tabular}
	\label{table:eigenvalue-error-dirichlet-subsonic}
\end{table}

\begin{figure}[H]
	\centering
	\includegraphics[width=0.7\linewidth]{figures/fixed-fixed-constant-v-v0=0.5}
	\caption{Showing the first 5 eigenvalues of each method in each case. All methods are close to the exact eigenvalues. These modes are stable.}
	\label{fig:constant-v-dirichlet-subsonic}
\end{figure}


\subsubsection*{Fixed-Open Boundary}
The numerical results agree with the exact solutions.
\begin{table} [H]
	\centering
	\caption{Relative error of each eigenvalue. Notice that the mode index starts from 0. These results agree with theory.}
	\begin{tabular}{|c|c|c|c|c|c|}
		\hline
		$v_0=0.5$ & 0         & 1         & 2         & 3         & 4         \\
		\hline
		FD        & 1.209e-05 & 3.458e-05 & 5.775e-05 & 8.153e-05 & 1.061e-04 \\
		\hline
		FE        & 8.090e-05 & 2.007e-04 & 2.981e-04 & 6.596e-04 & 1.821e-03 \\
		\hline
	\end{tabular}
	\label{table:eigenvalue-error-fixed-opened-subsonic}
\end{table}

\begin{figure}[H]
	\centering
	\includegraphics[width=0.7\linewidth]{figures/fixed-open-constant-v-v0=0.5}
	\caption{Showing the first 5 eigenvalues of each method. Finite-difference method has much better accuracy than finite-element method. All modes are stable.}
	\label{fig:constant-v-fixed-opened-subsonic}
\end{figure}


\subsection{Variable Velocity Profile}
\subsubsection*{Dirichlet Boundary}
When setting the mid-point velocity to be $M_m=0.5$, we have the subsonic velocity profile. This velocity profile is the orange line shown in Fig.~\ref{fig:velocity-profiles}. With Dirichlet boundary condition, $\tilde{v}(\pm 1) =0$. The flow in magnetic nozzle is stable. Fig.~\ref{fig:subsonic-v-dirichlet} shows the first few eigenvalues obtained by different discretizations.

The order of growth rates obtained by different methods is $10^{-13}$, we can consider it to be stable.
\begin{figure} [H]
	\centering
	\includegraphics[width=0.7\linewidth]{figures/fixed-fixed-subsonic-v}
	\caption{Showing the first 5 modes. It suggests that the flow in magnetic nozzle with subsonic velocity profile and Dirichlet boundary condition is stable.}
	\label{fig:subsonic-v-dirichlet}
\end{figure}

\subsubsection*{Fixed-Open Boundary}
\begin{figure} [H]
	\centering
	\includegraphics[width=0.7\linewidth]{figures/fixed-open-subsonic-v}
	\caption{Showing the first 5 modes. The ground mode is unstable, other modes are stable.}
	\label{fig:subsonic-v-fixed_open}
\end{figure}


\section{Supersonic Case}
\subsection{Constant Velocity Profile}
\subsubsection*{Dirichlet Boundary}
\begin{table} [H]
	\centering
	\caption{Relative error of each eigenvalue. Numerical results still agree with theory, but not as accurate as that under Dirichlet boundary.}
	\begin{tabular}{|c|c|c|c|c|c|}
		\hline
		$v_0=1.5$ & 1     & 2     & 3     & 4     & 5     \\
		\hline
		FD        & 0.001 & 0.005 & 0.010 & 0.019 & 0.030 \\
		\hline
		FE        & 0.006 & 0.010 & 0.019 & 0.029 & 0.043 \\
		\hline
		SE        & 0.001 & 0.005 & 0.011 & 0.019 & 0.030 \\
		\hline
	\end{tabular}
	\label{table:eigenvalue-error-dirichlet-supersonic}
\end{table}

\begin{figure}[H]
	\centering
	\includegraphics[width=0.7\linewidth]{figures/fixed-fixed-constant-v-v0=1.5}
	\caption{Showing the first 5 eigenvalues of each method in each case. All methods are close to the exact eigenvalues. Filtered modes are stable.}
	\label{fig:constant-v-dirichlet-supersonic}
\end{figure}

\subsubsection*{Fixed-Opened Boundary}
\begin{table} [H]
	\centering
	\caption{Relative error of each eigenvalue. Results agree with theory.}
	\begin{tabular}{|c|c|c|c|c|c|}
		\hline
		$v_0=1.5$ & 1         & 2         & 3         & 4         & 5         \\
		\hline
		FD        & 9.163e-05 & 2.435e-04 & 4.833e-04 & 8.160e-04 & 1.243e-03 \\
		\hline
		FE        & 4.431e-04 & 7.924e-04 & 1.516e-03 & 3.103e-03 & 8.001e-03 \\
		\hline
	\end{tabular}
	\label{table:eigenvalue-error-fixed-opened-supersonic}
\end{table}

\begin{figure}[H]
	\centering
	\includegraphics[width=0.7\linewidth]{figures/fixed-open-constant-v-v0=1.5}
	\caption{Showing the first 5 eigenvalues of each method. Finite-difference method has much better accuracy than finite-element method. All modes are unstable.}
	\label{fig:constant-v-fixed-opened-supersonic}
\end{figure}

\subsection{Variable Velocity Profile}
\subsubsection*{Dirichlet Boundary}
When the velocity profile is supersonic, shown as purple line in Fig.~\ref{fig:velocity-profiles}, spurious modes appeared as predicted in Chap.~\ref{chap:theoretical-analysis}. Using the convergence test, we successfully eliminate all unstable modes. Fig.~\ref{fig:supersonic-v-dirichlet} shows the first few filtered eigenvalues. As we can see the flow is stable.
\begin{figure} [H]
	\centering
	\includegraphics[width=0.7\linewidth]{figures/fixed-fixed-supersonic-v}
	\caption{First few filtered eigenvalues are shown. The spurious modes are filtered by convergence test.}
	\label{fig:supersonic-v-dirichlet}
\end{figure}

\subsubsection*{Fixed-Open Boundary}
All modes are unstable.
\begin{figure} [H]
	\centering
	\includegraphics[width=0.7\linewidth]{figures/fixed-open-supersonic-v}
	\caption{All modes are unstable.}
	\label{fig:supersonic-v-fixed-open}
\end{figure}


\section{Accelerating Case}
Starting from the singular point, we shoot the solution to the left boundary. We find the set of eigenvalues such that $\tilde{v}(-1)=0$. With these eigenvalues, we can extend the solution to the supersonic region $(0,1]$. The first few eigenmodes are drawn in Fig.~\ref{fig:results-accelerating-v}.
\begin{figure} [H]
	\centering
	\includegraphics[width=0.7\linewidth]{figures/results-accelerating-v}
	\caption{The eigenmodes are stable.}
	\label{fig:results-accelerating-v}
\end{figure}

\section{Decelerating Case}
We set the perturbations at the entrance of the nozzle to 0, $\tilde{v}(-1)=0$. Then we apply the same procedure as accelerating case, we obtained the following Fig.~\ref{fig:results-decelerating-v}. We are unable to obtain meaningful results, suggesting that the decelerating case might not be physically possible.
\begin{figure} [H]
	\centering
	\includegraphics[width=0.7\linewidth]{figures/results-decelerating-v}
	\caption{Unable to obtain meaningful results. The decelerating case might not be physically.}
	\label{fig:results-decelerating-v}
\end{figure}
