\chapter{Verification of Analytical Solutions}
The general solution to
\[
    \omega^2\tilde{v} + 2i\omega\pdv{\tilde{v}}{z} + (1-v_0^2)\pdv[2]{\tilde{v}}{z} = 0
\]
is
\[
    \tilde{v} = \exp(-\frac{i\omega}{v_0+1})
    \left[ \exp(i\omega\frac{z+1}{v_0+1})
        - \exp(i\omega\frac{z+1}{v_0-1}) \right]
\]
To show this, we first find the derivatives of $\tilde{v}$,
\begin{align*}
    \tilde{v}             & = \exp(-\frac{i\omega}{v_0+1})
    \left[ \exp(i\omega\frac{z+1}{v_0+1})
    - \exp(i\omega\frac{z+1}{v_0-1}) \right]                                                                     \\
    \pdv{\tilde{v}}{z}    & = i\omega\exp(-\frac{i\omega}{v_0+1})
    \left[ \frac{1}{v_0+1}\exp(i\omega\frac{z+1}{v_0+1}) - \frac{1}{v_0-1}\exp(i\omega\frac{z+1}{v_0-1}) \right] \\
    \pdv[2]{\tilde{v}}{z} & = -\omega^2\exp(-\frac{i\omega}{v_0+1})
    \left[ \frac{1}{(v_0+1)^2}\exp(i\omega\frac{z+1}{v_0+1}) - \frac{1}{(v_0-1)^2}\exp(i\omega\frac{z+1}{v_0-1}) \right]
\end{align*}
Then the rest is easy,
\begin{align*}
      & \omega^2\tilde{v} + 2i\omega\pdv{\tilde{v}}{z} + (1-v_0^2)\pdv[2]{\tilde{v}}{z}                                            \\
    = & \exp(-\frac{i\omega}{v_0+1})
    \left(1-\frac{2v_0}{v_0+1} + \frac{(1-v_0^2)}{(v_0+1)^2}\right)\exp(i\omega\frac{z+1}{v_0+1})
    \\
      & -\exp(-\frac{i\omega}{v_0+1})\left(1-\frac{2v_0}{v_0-1} + \frac{(1-v_0^2)}{(v_0-1)^2}\right)\exp(i\omega\frac{z+1}{v_0-1}) \\
    \\
    = & 0
\end{align*}

If $\omega = n\pi(1-v_0^2)/2$, then $\tilde{v}(\pm 1) = 0$.
It is easy to see that $v(-1)=0$. As for $z=1$, we have
\begin{align*}
    \tilde{v}(1) & \propto
    \exp(\frac{2i\omega}{v_0+1}) - \exp(\frac{2i\omega}{v_0-1}) \\
                 & =
    \exp(in\pi(1-v_0)) - \exp(-in\pi(1+v_0))                    \\
                 & =
    (-1)^n\exp(-in\pi v_0) - (-1)^n\exp(-in\pi v_0)             \\
                 & = 0
\end{align*}

If
\[\omega = (v_0^2-1) \left[ \frac{n\pi}{2} - \frac{1}{4}i\ln\left(\frac{v_0-1}{v_0+1}\right) \right]\]
then $\tilde{v}(-1) = 0$ and $\partial_z\tilde{v}(1) = 0$.
It is easy to see that $v(-1)=0$. The derivative at $z=1$ is
\begin{align*}
    \eval{\pdv{\tilde{v}}{z}}_{z=1} \propto &
    \frac{1}{v_0+1}\exp(\frac{2i\omega}{v_0+1}) - \frac{1}{v_0-1}\exp(\frac{2i\omega}{v_0-1})                                   \\
    =                                       & \frac{1}{v_0+1}\exp(in\pi(v_0-1) + \frac{v_0-1}{2}\ln(\frac{v_0-1}{v_0+1}))       \\
                                            & - \frac{1}{v_0-1}\exp(in\pi(v_0+1) + \frac{v_0+1}{2}\ln(\frac{v_0-1}{v_0+1}))     \\
    =                                       & \frac{(-1)^n}{v_0+1}\exp(in\pi v_0)\left(\frac{v_0-1}{v_0+1}\right)^{(v_0-1)/2}   \\
                                            & - \frac{(-1)^n}{v_0-1}\exp(in\pi v_0)\left(\frac{v_0-1}{v_0+1}\right)^{(v_0+1)/2} \\
    =                                       & 0
\end{align*}
The last equality holds because
\[ \frac{1}{v_0-1}\left(\frac{v_0-1}{v_0+1}\right)^{(v_0+1)/2}
    = \frac{1}{v_0+1}\left(\frac{v_0-1}{v_0+1}\right)^{(v_0-1)/2}  \]